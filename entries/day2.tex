\subsection{DAY 2 - Jumat, 30 Januari 2026}
Hari kedua di Pendidikan RSC Aksantara 2026 melanjutkan materi yang dipelajari untuk mantepin materi kita. Untuk day 2 ini mempelajari tentang \textbf{Protokol MAVLink dan SITL}.

\paragraph{MAVLink (Micro Air Vehicle Link)\newline}
Pada sesi ini, kita mempelajari tentang MAVLink atau Micro Air Vehicle Link yaitu protokol komunikasi pesan open source yang sangat ringan karena hanya beberapa puluh byte. Ini penting karena jaringan nirkabel punya bandwidth yang terbatas. MAVLink ini sudah menjadi standar untuk di riset dan industri, dipakai oleh sistem autopilot besar seperti PX4 dan ArduPilot. Selain itu, MAVLink mendukung komunikasi dua arah yang efisien dan bisa mengintegrasikan wahana UAV ke dalam jaringan internet.
\paragraph{System ID dan Component ID\newline}
Setiap entitas yang bisa mengirim atau menerima pesan MAVLink harus punya identitas unik agar tidak tertukar. Suatu sistem MAVLink (Wahana/GCS) bisa mempunyai banyak komponen.
\subparagraph{System ID}
\begin{itemize}
    \item Merujuk pada identitas unik untuk satu sistem UAV.
    \item Rentang nilai 1 sampai 255.
    \item System ID 255 dialokasikan untuk GCS.
    \item Semua komponen dalam wahana yang sama mempunyai System ID yang sama.
\end{itemize}
\subparagraph{Component ID}
\begin{itemize}
    \item Merujuk pada identitas spesifik untuk setiap komponen atau subsistem di dalam suatu wahana.
    \item Sudah ada penomoran bawaan dari MAVLINK:
    \begin{itemize}
        \item FC: 1
        \item Kamera: 100-105
        \item Gimbal: 154
        \item CC: 191-194
    \end{itemize}
\end{itemize}

\subsection*{Messages}
MAVLink punya berbagai jenis pesan untuk pertukaran data antarkomponen dalam sistem UAV. Tiap pesan punya ID unik yakni \textbf{Message ID}. Untuk MAVLink v1 ada 255, untuk MAVLink v2 hingga 16 juta. \textbf{Dialect} kumpulan definisi pesan yang dapat disesuaikan dengan kebutuhan sistem tertentu.
\begin{table}[h!]
\centering
\caption{Daftar MAVLink Messages}
\begin{tabular}{|l|c|c|p{7cm}|}
\hline
Message Name & MSG ID & Tipe & Fungsi Utama \\ 
\hline
HEARTBEAT & 0 & STATE & Menandakan sistem aktif (dikirim setiap 1 detik). \\ 
\hline
SYS STATUS & 1 & STATE & Status kesehatan sensor dan daya baterai. \\ 
\hline
ATTITUDE & 30 & STATE & Mengirimkan data kemiringan wahana (Roll, Pitch, Yaw). \\ 
\hline
GLOBAL POSITION INT & 33 & STATE & Posisi global (Lintang, Bujur, Ketinggian). \\ 
\hline
COMMAND LONG & 76 & COMMAND & Mengirim perintah aksi seperti ARM atau TAKEOFF. \\ 
\hline
MISSION ITEM & 39 & COMMAND & Mengirim titik koordinat rute terbang (waypoint). \\ 
\hline
\end{tabular}
\end{table}

\subsection*{Pola Interaksi dengan MAVLink}
\subparagraph{Streaming (Telemetry)}
\begin{itemize}
    \item Dipakai untuk data yang harus dipantau terus-menerus
    \item Bersifat broadcast (satu arah) secara periodik dari wahana ke GCS
    \item Mengandalkan Message ID untuk menentukan jenis data tanpa alamat tujuan spesifik.
    \item Use cases: untuk pesan bertipe State Messages
    \begin{itemize}
        \item HEARTBEAT
        \item ATTITUDE
        \item GLOBAL POSITION INT
    \end{itemize}
\end{itemize}
\subparagraph{Microservices}
\begin{itemize}
    \item Dipakai untuk perintah kritis yang butuh konfirmasi balasan.
    \item Bersifat point-to-point (dua arah), mirip client-service.
    \item Menggunakan System ID dan Component ID agar perintah tidak salah sasaran
    \item Use cases: untuk pesan bertipe Command Messages
    \begin{itemize}
        \item Upload/Download waypoints untuk misi
        \item Membaca atau mengubah pengaturan internal (parameter) pada FC
        \item Instruksi seperti ARM/DISARM, TAKEOFF, atau LAND.
    \end{itemize}
\end{itemize}

\paragraph{Anatomi Paket MAVLink}

\begin{figure}[h]
  \centering
  \includegraphics[width=1\textwidth]{images/anatomi_mavlink.png}
  \caption{Anatomi Paket MAVLink}
\end{figure}

\begin{itemize}
    \item Setiap paket dimulai dengan STX (0XFD) sebagai penanda bahwa paket tersebut menggunakan protokol MAVLink.
    \item SYS ID (System ID) dan COMP ID (Component ID) untuk identitas.
    \item Message ID untuk jenis message yang dikirim
    \item Payload membungkus konten data utama
    \item Checksum memastikan paket tidak corrupt saat diterima.
    \begin{itemize}
        \item Menggunakan algoritma CRC-16 (ITU X.25) untuk menghasilkan kode verifikasi 2 byte (CKA dan CKB) menggunakan data dari LEN sampai PAYLOAD.
    \end{itemize}
    \item Signature, seperti namanya, adalah tanda tangan digital (SHA-256) untuk memastikan bahwa paket MAVLink yang diterima berasal dari pihak terpercaya.
    \begin{itemize}
        \item Mencegah orang orangan janggal memfabrikasi paket MAVLink dan meng-hack wahana.
    \end{itemize}
\end{itemize}

\paragraph{MAVLink Interfaces}
\begin{itemize}
    \item MAVProxy: CLI-Based GCS, sangat ringan dan stabil.
    \item MAVSDK: SDK (software development kit) bagi developer untuk membangun aplikasi berbasis MAVLink dengan bahasa tingkat tinggi (C++, Python, dsb)
    \item MAVROS: SDK (software development kit) bagi developer untuk membangun aplikasi berbasis MAVLink dengan bahasa tingkat tinggi (C++, Python, dsb) 
\end{itemize}

\subsection*{Software In The Loop (SITL)\newline}
Software In The Loop adalah metode untuk menjalankan dan menguji sistem autopilot wahana UAV tanpa menggunakan perangkat keras fisik. FC dan lingkungan (hukum fisika) disimulasikan oleh perangkat lunak. Flight stack seperti ArduPilot/PX4 dijalankan seperti program biasa di komputer. Sehingga SITL ini Sangat aman dan cocok untuk eksperimen awal dan memeriksa ketepatan implementasi logika navigasi karena tidak ada risiko kerusakan fisik akibat crash.

\subsection*{Arsitektur Sistem SITL}

\begin{figure}[h]
  \centering
  \includegraphics[width=1\textwidth]{images/arsitektur_sitl.png}
  \caption{Arsitektur Sistem SITL}
\end{figure}

\paragraph{Tools untuk SITL}
\begin{itemize}
    \item ArduPilot: Flight stack untuk mensimulasikan Flight Controller.
    \item Gazebo Harmonic: Simulasi fisika dan lingkungan serta dinamika wahana yang realistis.
    \item MAVProxy: Jembatan komunikasi antara pengguna dan flight stack.
\end{itemize}

Untuk materi Day 2 ini aku cukup menguras tenaga dan mental, tapi aku dapet ilmu yang daging banget. Trus ada Handson yang bikin aku belajar ekstra banget tapi seru, karena aku bisa langsung coba dan praktek untuk ngeliat simulasi drone mulai dari persiapannya, gimana mereka dikontrol untuk mapping, sampe gimana cara landingnya. Jujur di materi ini aku harus baca materinya berkali kali, sambil cari referensi dari dokumentasi atau web web lain buat belajar.
